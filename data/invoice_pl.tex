% Report for studies.
% useful stuff:
% \raggedleft = left aligned
% \raggedright = right aligned
% \centering = centered
\documentclass[a4paper,12pt,oneside]{mwart}
\usepackage[MeX]{polski} % polish national characters
\usepackage[latin2]{inputenc} % character encoding
\usepackage{array}
%\usepackage{showframe}
\usepackage{color}
\usepackage[usenames,dvipsnames,svgnames,table]{xcolor}
\usepackage[cm]{fullpage}
\usepackage{changepage}
\usepackage[normalem]{ulem}
\usepackage{colortbl}
\usepackage{tabularx}
\usepackage{indentfirst}
\usepackage{graphicx}
\usepackage{arydshln}
\makeatletter
\makeatother
\pagestyle{empty}
\brokenpenalty=10000 % do not divide words across pages
\clubpenalty=10000 % do not leave orphants
\widowpenalty=10000 % do not leave widows
%\sloppy % do not extend lines if not fit
\frenchspacing
\hyphenation{FreeBSD UNIX Debian PostgreSQL MySQL PLD MSSQL perl python
Visual Studio Stable RedHat make gprof}
\renewcommand*{\familydefault}{\sfdefault}
\newcommand{\invoiceNo}{@invoiceNo@}
\newcommand{\invoiceDate}{@invoiceDate@}
\newcommand{\transactionDate}{@transactionDate@}
\newcommand{\dueDate}{@dueDate@}
\newcommand{\payMethod}{@payMethod@}
\newcommand{\documentType}{@documentType@}
\newcommand{\vendorName}{@vendorName@}
\newcommand{\vendorStreet}{@vendorStreet@}
\newcommand{\vendorCity}{@vendorCity@}
\newcommand{\vendorTIN}{@vendorTIN@}
\newcommand{\vendorExtra}{@vendorExtra@}
\newcommand{\vendeeName}{@vendeeName@}
\newcommand{\vendeeStreet}{@vendeeStreet@}
\newcommand{\vendeeCity}{@vendeeCity@}
\newcommand{\vendeeTIN}{@vendeeTIN@}
\newcommand{\vendeeExtra}{@vendeeExtra@}
\newcommand{\issuer}{@issuer@}
\newcommand{\netto}{@netto@}
\newcommand{\taxes}{@taxes@}
\newcommand{\items}{@items@}
\newcommand{\vatAmount}{@vatAmount@}
\newcommand{\brutto}{@brutto@}
\newcommand{\amountInWords}{@amountInWords@}
\newcommand{\amok}{\fontencoding{T1}\fontfamily{amk}\selectfont}
\newcommand{\sig}{\@signature@{\Large{\color{Blue!60!black}{\issuer}}}}
\begin{document}
\noindent
\begin{flushright}
	{\Large\textbf{Faktura VAT nr \invoiceNo}} \\
	\medskip
	\begin{tabular}{ l l }
		Data wystawienia: & \invoiceDate \\
		Data sprzeda�y: & \transactionDate \\
		Termin p�atno�ci: & \dueDate \\
		Metoda p�atno�ci: & \payMethod \\
	\end{tabular}
\end{flushright}
\medskip
\begin{adjustwidth}{-2cm}{-2cm}{}
	\begin{center}
		\documentType \\
		\bigskip
		\begin{tabular}{ c c c }
			\framebox{
				\parbox[t][4.5cm]{8.5cm}{
					\textbf{\textit{\uline{Sprzedawca}}}
					\smallskip \\
					\vendorName \\
					\vendorStreet \\
					\vendorCity \\
					\\
					\vendorTIN \\
					\vendorExtra
				}
			}
			&
			\framebox{
				\parbox[t][4.5cm]{8.5cm}{
					\textbf{\textit{\uline{Nabywca}}}
					\smallskip \\
					\vendeeName \\
					\vendeeStreet \\
					\vendeeCity \\
					\\
					\vendeeTIN \\
					\vendeeExtra
				}
			}
		\end{tabular}
		\bigskip
		\begin{footnotesize}
			\def\arraystretch{1.2}
			\begin{tabular}[b]{|r p{6cm} l r r r r r r|}
				\hline
					\rowcolor[rgb]{.8,.8,.8}
					\multicolumn{1}{|c|}{\textbf{Lp}}&
					\multicolumn{1}{c|}{\textbf{Nazwa towaru lub us�ugi}}&
					\multicolumn{1}{c|}{\textbf{Jedn}}&
					\multicolumn{1}{c|}{\textbf{Ilo��}}&
					\multicolumn{1}{c|}{\textbf{Cena}}&
					\multicolumn{1}{c|}{\parbox{1cm}{\centering \textbf{Stawka VAT}}}&
					\multicolumn{1}{c|}{\textbf{Netto}}&
					\multicolumn{1}{c|}{\parbox{1cm}{\centering \textbf{Kwota VAT}}}&
					\multicolumn{1}{c|}{\textbf{Brutto}} \\
				\hline
					\items
				\hline
			\end{tabular}
		\end{footnotesize}
	\end{center}
\end{adjustwidth}

\begin{footnotesize}
	\bigskip
	\begin{flushright}
		\def\arraystretch{1.2}
		\begin{tabular}{| r | r | r | r |}
			\hline
				\textbf{Stawka VAT}&\textbf{Warto�� netto}&\textbf{Kwota VAT}&\textbf{Warto�� brutto} \\
			\hline
				\taxes
			\hline
				\multicolumn{1}{|c|}{\textbf{Razem}}&\textbf{\netto}&\textbf{\vatAmount}&\textbf{\brutto} \\
			\hline
		\end{tabular}
	\end{flushright}
\end{footnotesize}

\bigskip
\begin{flushleft}
	\begin{small}
		\begin{tabular}{ l l }
			\textbf{Do zap�aty:}&\textbf{\brutto} \\
		\hline
			\textbf{S�ownie:}&\textbf{\amountInWords}
		\end{tabular}
	\end{small}
\end{flushleft}

\bigskip
\begin{tabular}{ c }
\sig \\
\hline
Wystawiaj�cy faktur�
\end{tabular}
\end{document}

